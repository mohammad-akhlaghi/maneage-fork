%% Copyright (C) 2018-2023 Mohammad Akhlaghi <mohammad@akhlaghi.org>
%% See the end of the file for license conditions.
\documentclass[preprint2, times, twocolappendix]{aastex631}

%% (OPTIONAL) CONVENIENCE VARIABLE: Only relevant when you use Maneage's
%% '\includetikz' macro to build plots/figures within LaTeX using TikZ or
%% PGFPlots. If so, when the Figure files (PDFs) are already built, you can
%% avoid TikZ or PGFPlots completely by commenting/removing the definition
%% of '\makepdf' below. This is useful when you don't want to slow-down a
%% LaTeX-only build of the project (for example this happens when you run
%% './project make dist'). See the definition of '\includetikz' in
%% 'tex/preamble-pgfplots.tex' for more.
\newcommand{\makepdf}{}

%% VALUES FROM ANALYSIS (NUMBERS AND STRINGS): this file is automatically
%% generated at the end of the processing and includes LaTeX macros
%% (defined with '\newcommand') for various processing outputs to be used
%% within the paper.
\input{tex/build/macros/project.tex}

%% MANEAGE-ONLY PREAMBLE: this file contains LaTeX constructs that are
%% provided by Maneage (for example enabling or disabling of highlighting
%% from the './project' script). They are not style-related.
\input{tex/src/preamble-maneage.tex}

%% PROJECT-SPECIFIC PREAMBLE: This is where you can include any LaTeX
%% setting for customizing your project.
%% Necessary macros for this project.
%%
%% These are a set of packages that have been commonly necessary in most
%% LaTeX usages. However, if any are not needed in your work, please feel
%% free to remove them.
%
%% Copyright (C) 2018-2023 Mohammad Akhlaghi <mohammad@akhlaghi.org>
%% Copyright (C) YYYY Your Name <your@email.address>
%
%% This file is free software: you can redistribute it and/or modify it
%% under the terms of the GNU General Public License as published by the
%% Free Software Foundation, either version 3 of the License, or (at your
%% option) any later version.
%
%% This file is distributed in the hope that it will be useful, but WITHOUT
%% ANY WARRANTY; without even the implied warranty of MERCHANTABILITY or
%% FITNESS FOR A PARTICULAR PURPOSE.  See the GNU General Public License
%% for more details.
%
%% You should have received a copy of the GNU General Public License along
%% with this file.  If not, see <http://www.gnu.org/licenses/>.





%% Packages you may need in your project
%% -------------------------------------
%
%% Here you can add/remove any custom LaTeX package that you need for this
%% project that aren't provided by the journal's style.

%% For loading images into the output (with '\includegraphics').
\usepackage{graphicx}

%% Color management.
\usepackage{xcolor}
\color{black} % Color of main text.
\definecolor{DarkBlue}{RGB}{0,0,90}

%% Manage links in the produced paper (for example their colors), and
%% include document information in the "Properties" of the PDF.
\hypersetup{
    pdftitle={\projecttitle},
    pdfauthor={\projectcopyrightowner},
    pdfsubject={\projectgitrepo{} (commit \projectversion)},
    pdfkeywords={Reproducible research, Maneage, ADD YOUR OWN}
}





%% PGFPlots templates
%% ------------------
%
%% These are ready-made customizations of these two commonly used packages
%% that you can use as a template for your own project: BibLaTeX (advanced
%% bibliography management) or PGFPlots (for drawing plots within LaTeX
%% directly from tables of data). If you don't use them, you can just
%% delete these two lines and also delete their files from your branch (to
%% keep the 'tex/src' directory on your branch clean).
\input{tex/src/preamble-pgfplots.tex}





%% AASTeX settings
\shorttitle{\projecttitle}
\shortauthors{Akhlaghi, M}











%% Start creating the paper.
\begin{document}

%% Title
\title{\projecttitle}

%% Authors
\author[0000-0012-3245-1234]{Your Name}
\affiliation{Your affiliation; \href{mailto:your@email.address}{your@email.address}}

%% Abstract
\begin{abstract}
  \noindent
  The abstract goes here.
  The figures of this research note are reproducible with Maneage, on the Git commit \projectversion.
\end{abstract}

%% Keywords (from https://astrothesaurus.org)
\keywords{Keyword1 name (XXXX), Keyword2 name (XXXX)}










%% Start of main body.
\section{Introduction}\label{sec:intro}
\noindent
The introduction goes here.

\section{Analysis}\label{sec:analysis}

You can use Figures like Figure \ref{fig:image-histogram}.

\begin{figure}[t]
  \includetikz{delete-me-image-histogram}{width=\linewidth}

  \caption{\label{fig:image-histogram} (a) An example image of the Wide-Field Planetary Camera 2, on board the Hubble Space Telescope from 1993 to 2009.
    This is one of the sample images from the FITS standard webpage, kept as examples for this file format.
    (b) Histogram of pixel values in (a).}
\end{figure}



\section{Acknowledgement}
The workflow of this research note was developed in the reproducible framework of Maneage \citep[\emph{Man}aging data lin\emph{eage},][latest Maneage commit \maneageversion{}, from \maneagedate]{maneage}.
This note is created from the Git commit {\projectversion} that is hosted on Codeberg\footnote{\url{\projectgitrepo}} and is archived on SoftwareHeritage for longevity.

The analysis of this research note was done using GNU Astronomy Utilities (Gnuastro, ascl.net/1801.009) version \gnuastroversion. Work on Gnuastro has been funded by the Japanese Ministry of Education, Culture, Sports, Science, and Technology (MEXT) scholarship and its Grant-in-Aid for Scientific Research (21244012, 24253003), the European Research Council (ERC) advanced grant 339659-MUSICOS, the Spanish Ministry of Economy and Competitiveness (MINECO, grant number AYA2016-76219-P) and the NextGenerationEU grant through the Recovery and Resilience Facility project ICTS-MRR-2021-03-CEFCA.

%% Bibliography
\bibliography{references}{}
\bibliographystyle{aasjournal}

%% Finish LaTeX
\end{document}

%% This file is part of Maneage (https://maneage.org).
%
%% This file is free software: you can redistribute it and/or modify it
%% under the terms of the GNU General Public License as published by the
%% Free Software Foundation, either version 3 of the License, or (at your
%% option) any later version.
%
%% This file is distributed in the hope that it will be useful, but WITHOUT
%% ANY WARRANTY; without even the implied warranty of MERCHANTABILITY or
%% FITNESS FOR A PARTICULAR PURPOSE.  See the GNU General Public License
%% for more details.
%
%% You should have received a copy of the GNU General Public License along
%% with this file.  If not, see <http://www.gnu.org/licenses/>.
